

\chapter{Introduction}
\label{chapter:introduction}

Modern software applications are large and complex. In addition, they run in
diverse environments and get inputs from variety of data sources. As a result,
predicting safe behavior for them at runtime can be difficult.
Moreover, giving assurances about the quality of service becomes practically
impossible when the applications are developed using third party libraries and
components. Such applications often exhibit vulnerabilities that can be
exploited by providing malicious inputs. In addition, diverse data sources and
complex constraints on them make it challenging for programmers to ensure that
all data elements are correctly validated and processed. Any deficiencies in
this process makes applications prone to failures. Such defects can be hard to
detect at the compilation-time, and irrespective of the validation techniques
used, some of them may go undetected and exist in the applications even when the
applications are in production.

The cost of failures can vary considerably depending on the mission-critical
nature of applications. In particular, it would be extremely undesirable for an
unmanned aerial vehicle on its mission to allow the control system to crash in
case of a failure. Instead a suboptimal functioning for a short while might be
acceptable until the system fully stabilizes.
Generally speaking, expectations about the quality of service would largely
depend on how a failure may impact the business. For example, a commercial
online store may not afford a crash while it is listing its products to a
customer. Such unpleasant experiences might result in customers moving to other
online stores making an adverse impact on the business. Similarly, a software
company launching a new product would expect it to be stable while the product
is undergoing beta-testing. Any crashes occurring at that time would result in
negative feedback from the users and loss in the business. To make the matter
worse, these failures may occur in software components that do not possess
critical functionality, and hence, may even get less attention to their quality
at the time of development. Nevertheless, irrespective of the criticality of
these components, if the crash occurs it is equally undesirable.

\lstset{language=Java , caption=Apache Log4j bug example.,
label=snippet:exampleRepairing1} \begin{figure}[t]
\begin{lstlisting}
private int substitute() {
  if (priorVariables == null) {
    priorVariables = new ArrayList<String>();
    priorVariables.add(new String(chars, offset, length));
  }
}
\end{lstlisting}
\end{figure}

As an example, consider Code ~\ref{snippet:exampleRepairing1} which depicts a
bug that
existed in Apache Log4j library version 2.0-beta9~\cite{ApacheLog4jBug} and crashed
the logging framework. It was reported as a major bug in
spite of the fact that it occurred in logging component. The object
\code{priorVariables} is a \code{List} of String. On line 4, there is no check on the variables to ensure
that invariants such as \code{offset + length <= chars.length}, \code{offset > 0}, and \code{length
> 0} hold. In case of a failure, rather than allowing the application to crash, organizations would
like to collect diagnostic information to identify the defects and allow the system to run suboptimal behavior
for a while until it stabilizes. As long as such suboptimal behavior is within acceptable
limits, the program survival would get higher preference. The bug can then be
fixed in the later versions.

Several approaches have been proposed in the past to ensure that programs can
recover from failures. Some of the approaches are based on static repairing
where the patches are synthesized automatically based on the counter examples
found in the field \cite{wei-issta-2010}.
However, it is not always desirable to shut down the system for the post-mortem analysis and then relaunch it
after fixing the defect. In order to overcome this weakness,
dynamic approaches have been proposed to deal with problems that are
related to memory, data, and incorrect programming constructs such as infinite
loops \cite{Carbin:2011, KlingMCR12, conf/sosp/PerkinsKLABCPSSSWZER09}. Some of the approaches work either by identifying and isolating
damaged data or memory portions \cite{conf/issre/DemskyR03, conf/icse/DemskyR05, conf/issta/DemskyEGMPR06}, or by delaying the execution until the program
self-stabilizes \cite{Eom:2012}, or by finding the alternative execution paths \cite{PezzeRWZ11}, or by disabling suppressing signals and hoping that the
program can recover automatically from the errors \cite{conf/pldi/LongSR14}. Static approaches strive
for correctness whereas dynamic approaches are typically optimistic and work on the
assumption that some suboptimal behavior under certain conditions is acceptable.

In this work, we propose a novel approach, which is hybrid in the nature and
deals with the failures originated from either malformed strings or incorrect
handling of strings. The approach first identifies program statements statically
that might be vulnerable to string-related failures, and then develops patches
by trying to identify origins of errors and constraints on the strings. It uses dynamic
analysis to improve the precision of the patches generated by the static
analysis. The approach targets string variables for patching, firstly, because
strings are used heavily in \java\ APIs and have been common sources
of errors, and secondly, because by targeting a specific type of data
the approach can develop patches that are more precise and result in the
behavior that would be close to the intended behavior. %We applied \tool\ to $x$
%hugely popular open-source libraries involving over $y$ million lines of
%code to patch $z$ high priority bugs resulting from unhandled runtime
%exceptions from \java\ \code{String} APIs.

This work makes following contributions:
\begin{itemize}

\item We present the design and implementation of \tool\
(\xref{chapter:motivation}, \xref{chapter:design} and
\xref{chapter:implementation}) that generates effective program patches to handle string-related errors. These 
patches get activated only in case of program failures during runtime, and save
program from crashing ensuring its acceptable behavior.

%by identifying error sources and solving constraints on string data mainly statically and
%partly dynamically. Since the technique takes into account constraints on
%the data, the generated patches have a higher chance of producing intended
%behavior.

% \item We present \tool\ (\xref{sec:implementation}), a tool that implements our approach
% and effectively generates program patches to handle string-related errors
% \cite{}.

\item We use a finite state machine (FSM) as a formalism (\xref{chapter:design})
to describe the behavior of \java\ \code{String} API, and apply it to drive the
generation of exception-specific patches.

\item We applied \tool\ to several hugely popular open-source libraries to patch
$30$ bugs, several of them rated critical or major, resulting from unhandled
runtime exceptions from \java\ \code{String} APIs. The results of our study
(\xref{chapter:results}) indicate that \tool\ can effectively produce patches
that save programs from crashing due to failures originating from known bugs. The
study also gives insights into the characteristics of the commonly occurring
string problems.

\item Manual inspection of the \tool-generated patches reveals that in most
cases they are semantically similar to the ones produced by the developers in
the later versions. %In fact, on couple of occasions the patches generated by
%\tool\ were found to be more accurate.
Thus, \tool\ can also guide developers in the process of building patches for
the future versions.
\end{itemize}

% The organization of the paper is as follows. Section~\ref{sec:motivation}
% describes the problem and provides an example to illustrate the proposed
% approach. Section~\ref{sec:architecture} provides the architecture of \tool\ and
% describes the analyses implemented by the tool. Section~\ref{sec:implementation}
% provides implementation details and Section~\ref{sec:evaluation} presents the
% results of the study. Finally, Section~\ref{ref:related work} presents the
% related work and Section~\ref{sec:conclusion} concludes with the outline of the
% future work.

We have made our source code and data sets available to the open source
community at \url{http://goo.gl/d1zcXD}. Github development repository is
located at \url{https://github.com/aritradhar/CLOTHO} and
\url{https://github.com/aritradhar/CLOTHO-TaintAnalysis}.

