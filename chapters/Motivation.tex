
\chapter{Motivation and Challenges}
\label{chapter:motivation}

Ensuring correctness of a program is an undecidable problem. In order to achieve
high scalability static analysis works by making sound approximations, which
typically lead to numerous false positives. Complex programming logic and data
coming from diverse sources make the problem worse. As a result, successful
execution of a real application can never be guaranteed, and unexpected failures
may happen. These failures often result in runtime exceptions being thrown by
the applications that are generally not handled. The cost of these runtime
failures can vary depending on the criticality of the applications and can be
very high for mission-critical applications.

\begin{table}[t]
\centering
\begin{tabular}{|c|c|c|}
\hline
\textbf{Runtime Exception Type} & \textbf{Occurrences} & \textbf{Percentage}\\
\hline 

\code{NullPointerException} & $34912$ & $54.94\%$ \\ 
\hline
\code{ClassCastException} & $7504$ & $11.81\%$ \\ 
\hline
\code{IndexOutOfBoundsException} & $6637$ & $10.44\%$ \\ 
\hline
\code{SecurityException}  & $5818$ & $9.15\%$ \\  
\hline
\code{NoSuchElementException} & $2392$ & $3.76\%$ \\ 
\hline
\code{ArithmeticException} & $2338$ & $3.67\%$ \\ 
\hline
\code{ConcurrentModificationExceptio} & $1889$ & $2.97\%$ \\ 
\hline
\code{DOMException} & $1024$ & $1.61\%$ \\ 
\hline
\code{ArrayStoreException} & $279$ & $0.43\%$ \\ 
\hline
\code{MissingResourceException} & $277$ & $0.43\%$ \\ 
\hline
\code{BufferOverFlowException} & $161$ & $0.25\%$ \\ 
\hline
\code{NegativeArraySizeException} & $122$ & $0.19\%$ \\ 
\hline
\code{BufferUnderFlowException} & $66$ & $0.1\%$ \\ 
\hline
\code{LSException} & $64$ &  $0.1\%$ \\ 
\hline
\code{MalformedParameterizedTypeExce} & $38$ & $0.05\%$ \\ 
\hline
\code{CMMException}  & $8$ & $0.01\%$ \\ 
\hline
\code{FileSystemNotFoundException} & $6$ & $0.009\%$ \\ 
\hline
\code{NoSuchMechanismException} & $3$ & $0.0045\%$ \\ 
\hline
\code{MirroredTypesException} & $1$ & $0.0015\%$ \\ 
\hline

\end{tabular}
\caption{Prominent runtime exceptions from
\texttt{stackoverflow}~\cite{stackoverflow}.}
\label{tab:stackoverlow}
\end{table}

\java\ applications are typically built using libraries, and \code{String} APIs
are commonly used in third party libraries. Common and diverse usage of strings
in programs is a significant source of errors. We mined the repository of posts
on \texttt{stackoverflow}~\cite{stackoverflow} to understand common types of
exceptions thrown in case of failures. Table~\ref{tab:stackoverlow} lists the
most prominent exceptions with the share of higher than $5\%$. The second column
indicates the types of exceptions, whereas the third column indicates their
overall percentage share. We observe that strings can play a role in generating
all except \code{SecurityException}. This result coupled with the potentially
heavy cost of program crashes motivated us to develop a hybrid technique for
automatic repairing of \java\ programs for failures related to \code{String}
APIs.

\section{Overview}
\label{subsec:overview}

\lstset{language=Java, caption=Snippet from \code{fileUtils} class of Apache
Commons library. , label = snippet:exampleRepairing2}
\begin{figure}[t]
\centering
\begin{lstlisting}
public static String getPathNoEndSeparator
        (String filename) {
  return doGetPath(filename, 0);
}
private static String doGetPath
        (String filename, int separatorAdd) {
  if(filename == null) return null;
  int prefix = getPrefixLength(filename);
  if (prefix < 0) return null;
  int index = indexOfLastSeparator(filename);
  if ((prefix >= filename.length()) || (index < 0))
        return "";
  return filename.substring(prefix,
        index + separatorAdd);
}
\end{lstlisting}
\end{figure}

Code~\ref{snippet:exampleRepairing2} corresponds to methods from from
\code{fileUtils} class of Apache Common IO library. The method
\code{getPathNoEndSeparator()} throws a \code{StringIndexOutOfBounds} exception
on Windows OS, which originates from statement \code{return
filename.substring(prefix, index + separatorAdd)} on line 13 when the method is
called with parameter \code{"/foo.xml"}.  Here, the value of \code{prefix} as
returned by the method \code{getPrefixLength} is 1. It fails to satisfy the
constraint implied by the program condition \code{prefix <= index +
separatorAdd} for \code{substring} method which  ensures that \code{beginIndex}
cannot be greater than \code{endIndex}. As a result, the exception being thrown.


\lstset{language=Java, caption=Patch for \code{fileUtils} class
from Apache Commons library bug., label = snippet:exampleRepairing3, firstnumber
=13}
\begin{figure}[t]
\centering
\begin{lstlisting}
String temp = null;
try {
  temp = filename.substring(prefix, index + separatorAdd);
} catch(IndexOutOfBoundsException ex) {
  int length = filename.length;
  int t = index + separatorAdd;
  temp = filename.substring(
    getStart(prefix,t,length), getEnd(prefix,t,length));
}
return temp;
\end{lstlisting}
\end{figure}

A closer inspection of this code snippet shows that the string variable
\code{filename} invokes two methods, namely \code{length} and \code{substring}
on lines 11 and 13 respectively. \java\ \code{String} API documentation
specifies that \code{length} does not throw any runtime exceptions. The only
exception that this invoke statement can throw is when the receiver object
referenced by \code{filename} is \code{null}. However, the check on line 7
indicates that this situation would not arise. However, method \code{substring}
can throw \code{IndexOutOfBoundsException} exception that can potentially crash
the program. A good patch to handle this failure should take into account all of
these observations. 

Code~\ref{snippet:exampleRepairing3} presents the patch automatically generated
by \tool . This patch replaces the invoke statement on line 13 in
Code~\ref{snippet:exampleRepairing2}. The invoke statement is now wrapped inside
the \code{try} block and a \code{catch} corresponding to
\code{IndexOutOfBoundsException} is added on line 15. This ensures that
control passes to the catch block only when the exception is thrown. Line 20
shows two method calls namely \code{getStart} and \code{getEnd} that are
inserted by \tool. These methods, using the knowledge about the length of
\code{filename} acquired with the help of the code on line 17, compute legally
correct indexes required by \code{substring} method to satisfy the constraint
related to \code{beginIndex} and \code{endIndex}. Method \code{substring} now
can regenerate the substring ensuring that the method call would not fail. 

The actual patch provided by the developers is semantically similar to the one
developed by \tool\ and both versions of the program generate exactly the same
output. Similarly, the patch developed by \tool\ for the bug depicted in
Code~\ref{snippet:exampleRepairing1} is semantically similar to the actual one
provided by the developers and is presented in
Code~\ref{snippet:exampleRepairing4}. Here the object referenced by the string
variable \code{temp} is regenerated after adjusting the offset and ensuring that
the constraint represented by the program condition \code{offset <= length}
would never be violated.


\lstset{language=java, caption=Patch for the Apache Log4j bug.,
label = snippet:exampleRepairing4, firstnumber =4}
\begin{figure}[t]
\centering
\begin{lstlisting}
try {
    temp = new String(chars, offset, length);
} catch(StringIndexOutOfBoundsException ex) {
    int i = chars.length;
    temp = new String(chars,
        IndexRepair.getStart(offset, length, i),
            IndexRepair.getEnd(offset, length, i));
}
priorVariables.add(temp);
\end{lstlisting}
\end{figure}

We next present in detail the techniques and the algorithms used in our analysis
that can produce patches to regenerate string variables under more complex
scenarios. Our study presented in \xref{chapter:results} suggests that majority
of the string generation scenarios in practice are less complex.

\section{Historical Context}
\label{section:historicalContext}

In recent past, we have seen couple of disastrous failures of critical military
and civilian infrastructure system due to system crash/shutdown/restart which is
results of some very common runtime exceptions.

\begin{itemize}
  
  \item In USS Yorktown, complete failure in propulsion and navigation system by
  a simple divide-by-zero exception in flight deck database
  ($1998$)~\cite{Ship}.
  
  \item AT\&T telephone network failure causing by one faulty switch causing ATC
  commutation blackout.
  
  \item Air-Traffic Control System in LA Airport lost communication with all
  $400$ airplanes caused by a system crash triggered by $32$ bit integer
  overflow~\cite{LAATC}.
  
  \item Mars rover curiosity B-side computer memory overflow causing OS suspend
  and multiple restart.
  
  \item Trans-Siberian Gas Pipeline Explosion in 1982 by deliberate bugs in
  software controlled valves.
  
  \item Near-blackout of the national grid in Austria caused by faulty function
  call.
  
\end{itemize}

All of these incidents have one thing common, all of them were critical systems
where availability is the major requirement. Most of the systems are such
critical that in case of failure one can not simple shutdown and restart the
system like general client applications as it may results in loss of human lives
and/or massive amount of money.







