\begin{abstract}

Programs are susceptible to malformed data coming from untrusted sources.
Occasionally the programming logic or constructs used are inappropriate to
handle all types of constraints that are imposed by legal and well-formed data.
As a result programs produce unexpected results or even worse, they may crash.
Program behavior in both of these cases would be highly undesirable.

In this thesis work, we present a novel hybrid approach that saves programs from
crashing when the failures originate from malformed strings or inappropriate
handling of strings. Our approach statically analyses a program to identify
statements that are vulnerable to failures related to associated string data. It
then generates patches that are likely to satisfy constraints on the data, and
in case of failures produce program behavior which would be close to the
expected. The precision of the patches is improved with the help of a dynamic
analysis. The patches are activated only after a failure is detected, and the
technique incurs no runtime overhead during normal course of execution, and
negligible overhead in case of failures.

We have experimented with \java\ \code{String} API, and applied \tool\ to
several hugely popular open-source libraries to patch $30$ bugs, several of
them rated either critical or major.
% that a few commonly used \java\ APIs.
Our evaluation shows that \tool\ is both practical and effective. The comparison
of the patches generated by our technique with the actual patches developed by
the programmers in the later versions shows that they are semantically similar.

 

\end{abstract}