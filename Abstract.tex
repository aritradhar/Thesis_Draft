\begin{abstract}

Runtime Exceptions are common types of exceptions which may lead to system crash
which results in shutdown or restart. For may critical application such scenario
is unacceptable due to their nature which requires availability of the service.
%Runtime exceptions may even leads to inconsistency of data in the system which
% may leads to further complications and expensive fixes.

Program bugs which causes runtime exceptions often go unnoticed at the time of
development as these exceptions are unchecked exceptions. The key issue is to
guide the program through some exception suppression procedure which will leads
the program to a consistent state hence improve the chance of surviving a fatal
crash. Here we consider such programs for which restart is not an option.

In our work, we present a novel technique to recover from unexpected runtime
exceptions. We have used hybrid of two techniques for efficient detection of
potential point of failure and patch it closest to that to minimize the damage.
One technique uses type of runtime exception to apply appropriate patch. The
other technique will provides typestate analysis technique which will detect
typestate violations to apply the right patch. We also performed static analysis
to taint variables and object to make sure no patching is applied to those which
leaves the system. We only patched those statements for which the
associated variables and objects stay and die inside the program. The taint
analysis phase is follwed by the repairing phase where we tried to apply the
patch close to the point of exception thus minimizing the effect of exception.
%The developer will flag a specific portion of the program to be eligible for 
%repairing and patching.
 

\end{abstract}