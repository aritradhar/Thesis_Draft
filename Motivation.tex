
\chapter{Motivation and Challenges}
\label{chapter:motivation}


\section{Historical Context}
\label{section:historicalContext}

In recent past, we have seen couple of disastrous failure of critical military
and civilian infrastructure system due to system failure/crash which is results
of some very common runtime exceptions.

\begin{itemize}
  
  \item In USS Yorktown, complete failure in propulsion and navigation system by
  a simple divide-by-zero exception in flight deck database (1998)~\cite{Ship}.
  
  \item AT\&T telephone network failure causing by one faulty switch causing ATC
  commutation blackout.
  
  \item Air-Traffic Control System in LA Airport lost communication with all 400
  airplanes caused by a system crash triggered by integer (32bit)
  overflow~\cite{LAATC}.
  
  \item Mars rover curiosity B-side computer memory overflow causing OS suspend
  and multiple restart.
  
  \item Trans-Siberian Gas Pipeline Explosion in 1982 by deliberate bugs in
  software controlled valves.
  
  \item Near-blackout of the national grid in Austria caused by faulty function
  call.
  
\end{itemize}

All of these incidents have one thing common, all of them were critical system
where availability is the major requirement. Most of the systems are such
critical that in case of failure one can not simple shutdown and restart the
system like general client applications as it may results in loss of human lives
and massive amount of money.

\section{Data from Stack Overflow Posts}
\label{section:stackoverflow}

\begin{table}[h]
\centering
\begin{tabular}{|c|c|c|}
\hline
\textbf{Runtime Exception Type} & \textbf{Occurrences} & \textbf{Percentage}\\
\hline NullPointerException & $34912$ & $54.94\%$ \\ \hline

ClassCastException & $7504$ & $11.81\%$ \\ \hline

IndexOutOfBoundsException & $6637$ & $10.44\%$ \\ \hline

SecurityException  & $5818$ & $9.15\%$ \\  \hline

NoSuchElementException & $2392$ & $3.76\%$ \\ \hline

ArithmeticException & $2338$ & $3.67\%$ \\ \hline

ConcurrentModificationExceptio & $1889$ & $2.97\%$ \\ \hline

DOMException & $1024$ & $1.61\%$ \\ \hline

ArrayStoreException & $279$ & $0.43\%$ \\ \hline

MissingResourceException & $277$ & $0.43\%$ \\ \hline

BufferOverFlowException & $161$ & $0.25\%$ \\ \hline

NegativeArraySizeException & $122$ & $0.19\%$ \\ \hline

BufferUnderFlowException & $66$ & $0.1\%$ \\ \hline

LSException & $64$ &  $0.1\%$ \\ \hline

MalformedParameterizedTypeExce & $38$ & $0.05\%$ \\ \hline

CMMException  & $8$ & $0.01\%$ \\ \hline

FileSystemNotFoundException & $6$ & $0.009\%$ \\ \hline

NoSuchMechanismException & $3$ & $0.0045\%$ \\ \hline

MirroredTypesException & $1$ & $0.0015\%$ \\ \hline

\end{tabular}
\caption{Most frequent java runtime exceptions from stack overflow data}
\label{tab:stackoverlow}
\end{table}

We have analyzed data from stack overflow and we looked for java runtime
exception which are discussed most frequently. In the
table~\ref{tab:stackoverlow}, the data we find is tabulated along with their
occurrences and percentages.


From the table it is clear that null pointer exception in java is not only the
most frequent but also the most dominant runtime exception having share of more
than $50\%$. This data is highly motivational for us as there are number of
cases where Java developers encounters software bugs which are mostly based on
runtime exception.

\section{Major Challenges}
\label{section:challenges}

The challenges we faced during the research can be described in few points :

\begin{enumerate}
  \item The major challenge was that the program we try to patch, in all the
  cases we actually don't know the internal logic of the program, we have
  patched it based on its behavior which can be damaging to the system itself.
  To prevent this we have tainted input variables which are coming from out side
  environment and see how they are interacting with other variables and object
  in the program. In case any of these variables go outside of the system, we
  marked the path to make sure patch won't be applied along that path.
  
  \item Most of the patching are non-trivial in nature and adaptive based on the
  use case to take full advantages what resources available in the code rather
  than some deterministic patching technique.
\end{enumerate}


